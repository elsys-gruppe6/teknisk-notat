\section{Problemstilling}
\label{sec:problemstilling}
\textit{En relativt kort beskrivelse av den overordnede problemstillingen.}
\begin{itemize}
\item \textit{Hva er behovet}
\item \textit{Hvorfor er det viktig}
\item \textit{Hvem (hvilke grupper) vil kunne være interessert i er system som oppfyller ett eller flere behov innen den overordnede problemstillingen}
\end{itemize}


Det globale behovet for energi er stadig i vekst, og er i dag større enn noen gang. Den raske veksten i levestandard og befolkning som verden opplever i dag gjør at kraftproduksjonen og infrastruktur har vanskelig for å holde tritt. I dag løses dette hovedsakelig ved bruk av mindre\todo{Finne Kilde} fossile aggregat som plasseres der infrastrukturen ikke klarer å levere nok. Dette fører til økt utslipp av drivhusgasser og senker luftkvalitet i nærområdet. Det fossile brennstoffet trenger også å transporteres til disse avsidesliggende samfunnene, som både er dyrt og fører til mer utslipp. Gode alternativer for dieselaggregat er blant annet vindturbiner. Et eksempel fra Norge der man vurderer vindturbiner for å kunne tilfredsstille moderne strømforbruk er på øysamfunnet Froan ca 90km nordvest for Trondheim \cite{AvisFroanVind}.


Selv om vindturbiner er fornybare kommer de ikke helt uten konsekvenser. En av hovedbekymringene som rettes mot vindturbiner er at de påvirker den lokale naturen i stor grad. Hvert år dør millioner av fugler som følge av kollisjon med vindturbiner\cite{dodfugler}. Dette tallet er lite sammenliknet med andre energikilder \cite{dodfugler}, men dersom vindturbinene konstrueres i habitatet til en sjelden eller utrydningstruet fugleart blir konsekvensene mye større. Som et resultat av disse bekymringene er det et behov for måter å begrense omfanget av vindturbinenes skade på fugleliv. Men hvordan vet vi hvor mye skade en vindturbin kan gjøre på fuglelivet før den er satt opp? Og vet vi hvor stor påvirkning vindturbiner har på fuglelivet?


%Det globale behovet for energi er stadig i vekst, og er i dag større enn noen gang. Den raske veksten i levestandard og befolkning som verden opplever i dag gjør at kraftproduksjonen har vanskelig for å holde tritt. Ofte er de eneste praktiske energikildene svært skadelige for miljøet. I nyere tid har folk over hele verden blitt mer miljøbevisste, og fornybare kilder til energi blir mer og mer utbredt. Blant disse er vindturbiner. Disse har ikke den samme påvirkningen på miljøet som fossile kraftverk har, men de kommer ikke uten konsekvenser. En av hovedbekymringene som rettes mot vindturbiner er at de påvirker den lokale naturen i stor grad. Hvert år dør millioner av fugler [1] som følge av kollisjon med vindturbiner. Dette tallet er lite sammenliknet med andre energikilder [2], men dersom vindturbinene konstrueres i habitatet til en sjelden eller utrydningstruet fugleart blir konsekvensene mye større. Som et resultat av disse bekymringene er det et behov for måter å begrense omfanget av vindturbinenes skade på fugleliv.I Norge brukes vindturbiner for å gi enkelte avsidesliggende lokalsamfunn et eget lokalt strømnett framfor å koble dem til hovedstrømnettet, slik som på Froan [3] og Rye [4]. Her er vindturbiner en praktisk løsning, men fuglelivet må tas hensyn til. Derfor ønsker TrønderEnergi, som er ansvarlige for utbyggingen av disse turbinene, et produkt som kan bidra til å innhente informasjon om miljøpåvirkningen av turbinene eller begrense den. Systemet som utvikles skal være i stand til å overvåke fugleaktiviteten i et område. Denne dataen skal så presenteres på en nettside. Dette kan så brukes til å vurdere ulike områder hvor det planlegges vindturbiner, eller for å kartlegge påvirkningen fra turbiner som allerede er konstruert. Det kan også brukes andre steder hvor det er nødvendig å samle informasjon om fugleliv, slik som små lufthavner.
