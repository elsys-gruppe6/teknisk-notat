\section{Problemstilling}
\label{sec:problemstilling}
%\textit{En relativt kort beskrivelse av den overordnede problemstillingen.}
%\begin{itemize}
%\item \textit{Hva er behovet}
%\item \textit{Hvorfor er det viktig}
%\item \textit{Hvem (hvilke grupper) vil kunne være interessert i er system som oppfyller ett eller flere behov innen den overordnede problemstillingen}
%\end{itemize}


Behovet for energi er stadig i vekst, og er i dag større enn noen gang. 
Den raske veksten i levestandard og befolkning som verden opplever i dag, gjør at kraftproduksjon og infrastruktur har vanskelig for å holde tritt. 
Spesielt utsatt er små samfunn plassert langt unna eksisterende infrastruktur, der utvidelse eller oppgradering av eksisterende strømnett vil bli svært dyrt. 
En mulig løsning på dette problemet, er å gjøre det gjeldende samfunnet selvforsynt med \textit{off-grid} fornybar energi, for eksempel fra vindturbiner. 

Øysamfunnet Froan i Frøya kommune i Trøndelag, ca. 90 km nordvest for Trondheim, er et område hvor nettopp dette vurderes\cite{AvisFroanVind}. 
På 60-tallet ble det bygget en 23 km lang sjøkabel for å koble Froan til strømnettet. 
Denne ble så fornyet på 80-tallet, og har siden det forsynt de 115 kundene på øyene med strøm. 
Her ønsker TrønderEnergi å bygge ut vindturbiner, slik at øysamfunnet er selvforsynt med strøm innen kabelen må fornyes igjen. 
Et av problemene er at det finnes mye sårbart fugleliv i området, som hubro og havørn\cite{froandn}.

Selv om vindturbiner ikke slipper ut miljøgasser, utgjør de en potensiell risiko for dyrelivet i nærheten. 
Hvert år dør millioner av fugler som følge av kollisjon med en vindturbin\cite{dodfugler}. 
Dette tallet er imidlertid lite sammenliknet med andre energikilder\cite{dodfugler}.
Dersom vindturbinene installeres i habitatet til en sjelden eller utrydningstruet fugleart, kan dette likevel skape problemer.
Vindturbiner kan føre til en sterk svekking av en allerede truet art, og i verste fall bidra til utryddelse av arter. 
Direktoratet for naturforvaltning har dermed krevd en nøye utreding av fugleaktiviteten i området før en eventuell utbygging på Froan, slik at vindturbinene forstyrrer fuglelivet minst mulig \cite{froandn}. 
Det er dermed et stort behov for å kartlegge fugleaktivitet i området rundt eksisterende eller planlagte vindturbiner.
Dette gjøres i dag i stor grad manuelt av ornitologer, som gjør det til en svært kostbar prosess og begrenser mengden data man får samlet inn.

I samarbeid med TrønderEnergi, skal vi i Jolyu takle denne utfordringen. 
Ved å lage et automatisk overvåkingssystem som detekterer, teller og kartlegger fugleaktivitet i et område, muliggjøres innsamling av store mengder data fra det aktuelle området. 
Det vil bidra til at utbygging kan settes i gang raskere med redusert kostnad.