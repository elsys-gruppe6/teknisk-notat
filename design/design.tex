\section{Design}
\label{sec:design}
\textit{Her kommer detaljene om \textbf{hvordan} en har tenkt at konseptet skal kunne fungere. Det er fremdeles snakk om prinsipper og ideer, ikke fysisk realisering og komponentvalg.}


\subsection{Kamera}

Et infrarødt kamera (IR-kamera) brukes til å detektere infrarød stråling fra fugler. Infrarød stråling er elektromagnetisk stråling med lengre bølgelengde enn synlig lys, mellom 700nm og 1mm. Sensorer i dette spekteret kan detektere alle objekter med en temperatur over det absolutte nullpunkt, 0K (-273.15$\degree$C), og opp til 3864K (4137$\degree$C). Kroppstemperaturen til fugler ligger ved omtrent 40$\degree$C \cite{fugltemp}, og omgivelsene kan antas å holde mellom -15$\degree$C og 30$\degree$C utifra maks og minimum temperatur i Trondheim. Dette tilsvarer langbølge-IR med bølgelengder på $8-15\mu m$. Et IR-kamera bruker Stefan-Boltzmanns lov for å regne ut temperaturen til objektet: $\Theta=\epsilon \sigma T^4$, der $\Theta$ er varmestrålingseffekt per areal (W/$m^2$), T er temperatur i kelvin, $\sigma$ er Boltzmanns konstant og $\epsilon$ er den termiske emissiviteten til objektet som skal måles. For fugler er denne emissiviteten målt til å være ca. 0.95 \cite{fuglemm}.

\todo{Kanskje legge til emmisiviteten til himmelen}

\subsection{Deteksjon}

Prosesseringsenhetesn skal få inn en kontinuerlig strøm av bilder fra kameraet og behandle denne bilde for bilde. Bildebehandlingen skal finne objekter i bildet, eller såkalte ''blobs''. Objektene som skal oppdages er fugler, og disse vil skille seg ut fra bakgrunnen på grunn av deres høyere temperatur, som representeres ved at de har ulik farge fra bakgrunnen. Bildebehandlingen skal kunne detektere flere objekter i samme bilde. I tillegg skal den kunne se på bilde n og sammenlikne med bilde n-1 og finne ut om en blob har beveget seg. Slik unngås duplikater der en fugl telles to ganger.

I utgangspunktet skal produktet kunne detektere fugler ved en vindmølle som er 50 meter høy, med diameter på 27 meter. For å finne vinkelen kameraet trenger vil kan vi bruke trigonometri.
\begin{equation}
    2\tan(x)=\frac{36.5}{27}
\end{equation}
som gir minimumsvinkelen til kameraet vårt må være på 36.05 grader. \todo{Kanskje flytte til implementasjon}

\subsection{Værstasjon}
Systemet skal ha egne sensorer for innsamling av værdata. Dette vil være sensorer for måling av lufttrykk, lufttemperatur, luftfuktighet, vindstyrke, vindretning og regn. Dataene fra sensorene skal deretter knyttes opp mot fugleaktiviteten for å finne eventuelle sammenhenger i vær og fugleaktivitet. For eksempel, hvis all fugleaktivitet skjer når det ikke blåser, så vil kollisjon med vindturbiner være en mindre reell fare, da turbinene ikke beveger seg i slikt vær. 

\subsection{Strukturelt}

Prosesseringsenheten og kameraet skal dele en boks. Boksen skal være over bakkenivå for å unngå interferens fra dyr nære bakken, og for å ikke tildekkes av snø eller lav vegetasjon. Boksen printes i plastikk, og skal være værbestandig. 

\subsection{Nettside}

Systemet vil ha en nettside som skal kunne framstille dataene som samles fra kamera og værstasjonen. Måten dataen vises på skal være slik at den er lett forståelig og navigerbar for en bruker uten noen spesielle tekniske kunnskaper. Dataen skal kunne sorteres etter behov for å se trender i fugleaktivitet, for eksempel time for time eller dag for dag. 




------------

Systemet har en sentral prosesseringsenhet som styrer alle undersystemene som kamera, værstasjon og eksterntilkobling. 

Systemet skal ha en sentral prosesseringsenhet som styrer to undersystemer; kamera og værstasjonen. I tillegg skal denne ha eksterntilkobling til databasen og nettsiden, sammen med lokal lagring for backup. 

prosesseringsenheten vil motta en kontinuerlig strøm av video fra kameraet. Her vil hvert bilde bli behandlet internt for fugler. I tillegg skal bildene ses på kontinuerlig. Dvs om vi detekterer en fugl på bilde n og på bilde n+1 så er det en stor sannsynlighet for at dette er samme fugl. Systemet skal forholde seg til dette, og skal ved hjelp av tracking telle fuglen kun en gang, selv om fuglen er i videostrømmen i flere bilder.

Når fuglene er detektert skal informasjon om disse sendes til en database, der de skal ligge lagret. Dataene fra denne databasen skal vises på en nettside som gir et oversiktlig bilde over fugleaktiviteten rundt vindturbinen.

Tabell over systemkrav
% Please add the following required packages to your document preamble:
% \usepackage{graphicx}
\begin{table}[!htbp]
\centering
\resizebox{\textwidth}{!}{%
\begin{tabular}{|l|l|l|}
\hline
\multicolumn{3}{|l|}{\textbf{Generelt}} \\ \hline
\textbf{Kravnavn} & \textbf{Beskrivelse} & \textbf{id} \\ \hline
Mål & Systemet skal detektere og dokumetere fugleaktivitet i luften &  \\ \hline
Underlag & Omerådet vil ha et flatt underlag &  \\ \hline
Vegetasjon/miljø & Området vil være minst 15x8m med åpent areal &  \\ \hline
Toleranse/tetthet/IP-rating & Systemet skal minst være IP55W. &  \\ \hline
Strømforsyning & Systemet vil ha strømforsyning fra strømnettet &  \\ \hline
Sensorer & Systemet vil ha temperatur, luftfuktighet, trykk og vindsensor &  \\ \hline
\multicolumn{3}{|l|}{\textbf{Deteksjon}} \\ \hline
\textbf{Kravnavn} & \textbf{Beskrivelse} & \textbf{id} \\ \hline
Kamera & Systemet vil bruke et imfrarødt kamera med FOV minst $\ang{41}x\ang{31}$ og 9 Hz bildefrekvens &  \\ \hline
Rekkevidde & Systemet skal minst ha en rekkevidde på 50 meter &  \\ \hline
Temperatur & Systemet skal minst detektere objekter på $\ang{20}$C med størrelse 300x100mm ved 50m &  \\ \hline
\multicolumn{3}{|l|}{\textbf{Casing}} \\ \hline
\textbf{Kravnavn} & \textbf{Beskrivelse} & \textbf{} \\ \hline
Dimensjoner & Casingen vil være mindre enn 200*300*300mm &  \\ \hline
\multicolumn{3}{|l|}{\textbf{Overføring, behandling og fremstilling av data}} \\ \hline
\textbf{Kravnavn} & \textbf{Beskrivelse} & \textbf{} \\ \hline
Prosesseringsenhet & Dataene vil behandles av en Raspberry Pi. &  \\ \hline
Programvare & Programvaren vår vil være basert på åpen kildekode &  \\ \hline
Treffrate & Programvaren skal være god nok til å gi treffrate 75\%. &  \\ \hline
Overføring intert & Systemet vil bruke USB til å overføre data mellom kamera og prosesseringsenhet. &  \\ \hline
Overføring eksternt & Systemet vil bruke Wifi for å overføre data fra prosesseringsenhet og databasen. &  \\ \hline
\end{tabular}%
}
\end{table}
