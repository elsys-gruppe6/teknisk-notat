%Dokumentinnstillinger:---------------------------------
\documentclass[11pt,norsk]{elsys-teknisk}

\heading{Teknisk Notat} 
\title{Fugletitter}
\author{Forfatter Forfattersen}
\version{1.0}
\date{\today}

\begin{document}

\maketitle

%Automatisk generert innholdsfortegnelse:------------------
\toc

%Selve rapporten:------------------------------------------
\textit{Tekst i kursiv er ment som forklaring, tekst uten kursiv skal være med som den er. Kursivert tekst fjernes i den endelige rapporten.\\
\\
Denne malen er ment å gi tips for hvordan skrive en god teknisk rapport som dokumenterer et designet og testet elektronisk system. Den tiltenkte leser er en teknisk kyndig person som har bruk for dokumentasjonen i forbindelse med videreutvikling, vedlikehold, reparasjon eller redesign av systemet.\\
\\
Malen er inndelt i de samme kaptiler som bør være med i en teknisk rapport, og innholdet i kapitlene er beskrevet fortløpende.
}

\section{Problemstilling}
\label{sec:problemstilling}
\textit{En relativt kort beskrivelse av den overordnede problemstillingen.}
\begin{itemize}
\item \textit{Hva er behovet}
\item \textit{Hvorfor er det viktig}
\item \textit{Hvem (hvilke grupper) vil kunne være interessert i er system som oppfyller ett eller flere behov innen den overordnede problemstillingen}
\end{itemize}

\section{Konsept}
\label{sec:konsept}
\textit{En overordnet beskrivelse av \textbf{hva} systemet skal gjøre. Her legges vekt på hvordan systemet skal oppføre seg, ikke \textbf{hvordan} det er designet.}

\section{Design}
\label{sec:design}
\textit{Her kommer detaljene om \textbf{hvordan} en har tenkt at konseptet skal kunne fungere. Det er fremdeles snakk om prinsipper og ideer, ikke fysisk realisering og komponentvalg.}

\section{Implementering}
\label{sec:implementering}
\textit{Enda mer \textbf{hvordan} kommer her. Nå er vi på detaljnivå med detaljerte kretsskjema og dokumentasjon av algoritmer og kode. Større mengder kildekode er ikke aktuelt å ta med, men kortere snutter for å beskrive spesielle løsninger kan tas med. Resten av kildekoden legges ved som vedlegg. \\
\\
Detaljer puttes i vedlegg.}

\section{Verifikasjon og test}
\label{sec:verifikasjon}
\textit{Her dokumenteres hvordan systemet er testet. Resultat av test og drøfting av potensielle forbedringer. Det er viktig å få med at systemet eller deler av systemet virker eller ikke virker. Dersom det er mulig å tallfeste \textbf{hvor godt} systemet virker, er det bra.}

\section{Konklusjon og anbefalinger}
\label{sec:konklusjon}
\textit{Kort oppsummering av den innsikt som er oppnådd som er aktuelt for videreføring i Fase II. Enkelt-resultater som kan tallfestes bør være med.\\
\\
NB: Vær konkret på dette punktet. Det er uinteressant \textbf{at} gruppen har oppnådd innsikt på det ene eller andre området. Det interessante er \textbf{hvilken} innsikt som er oppnådd. \\
\\
Anbefalinger for videreføring, bruk og vedlikehold er også viktig å få med.}



%Bibliografi: Legg til flere elementer ved å legge til flere \bibitem:--------
\phantomsection
\addcontentsline{toc}{section}{Referanser}
\begin{thebibliography}{99}

\bibitem{bibelen}
  Albert Einstein,
  \emph{Elektronikkbibelen},
  O Store Forlag,
  1. utgave,
  1930.

\end{thebibliography}

\appendix

%Tillegg. Flere tillegg legges til ved å lage flere sections:-----------------
\section{Vedlegg 1}
\label{sec:vedlegg1}
\textit{Ikke-nummerert rapportdel der det kan legges stoff som kan være av aktuelt for spesielt interesserte lesere og som ville redusert lesbarheten om det hadde vært inkludert i hovedteksten. Eksempler kan være større tabeller eller fullstendig kildekode.}

\end{document}